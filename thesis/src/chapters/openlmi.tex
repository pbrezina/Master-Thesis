\chapter{OpenLMI}
\label{chapter:openlmi}

OpenLMI (Open Linux Management Infrastructure) is a set of tools that simplifies
monitoring and management of Linux machines through WBEM technology. Its goal is
to unify configuration of all fundamental parts of Linux system, thus reducing
the needs to understand the syntax and semantics of configuration files for
every service.

OpenLMI uses Open Pegasus as underlying WBEM implementation and adds several new
providers that cover many important areas of common administration tasks like
disk partitioning, network configuration, users and groups management, software
installation and maintenance, event log, hardware information, etc.

The project also provides several new client tools. Besides low level API and
language bindings for C/C++, Java and Python it brings especially the LMI shell
which is a client application that reflects all CIM objects into the Python
language. This enables full access to remote objects and allows to use them
natively in Python code. It also offers a set of LMI scripts that implements the
most common tasks and completely hides the complexity of WBEM under simple
commands.

\subsubsection{Main features}

\begin{itemize}
  \item Open source project with active development
  \item Linux-specific set of providers
  \item Python-based LMI shell
  \item Simplifies common task with single commands
  \item API for C/C++, Java and Python
\end{itemize}

\section{LMI Providers}
\label{openlmi:providers}

OpenLMI project aims at delivering CIM providers that implements classes for
monitoring and configuration of fundamental parts of Linux systems.
These classes can be simply recognized in the CIM schema since name of all
classes developed for this project starts with |LMI_| prefix.

OpenLMI version 0.4.2\footnotemark contains the following providers: network,
storage, fan, hardware, logicalfile, power, service, software, journald and
realmd.

\footnotetext{The current OpenLMI release at the time of this thesis.}

\section{LMI Shell}
\label{openlmi:shell}

The LMI shell is a client application for OpenLMI. It works as a Python
interpreter therefore it is capable of running any code written in this
language. All remote CIM objects are reflected into a native Python objects,
thus they can be simply accessed by using the language constructs.

As a consequence of this feature, the operator does not have to deal with the
complexity of WBEM communication mechanisms and data formats. The only knowledge
required to use the shell is an elementary Python syntax and basic programming
skills. However, the more fluent in the Python language the operator is the more
powerful the shell becomes.

The following section introduces some fundamental concepts of accessing OpenLMI
from the shell.

\subsection{Using the shell}
\label{openlmi:shell:examples}

This section contains set of example WBEM calls invoked from LMI shell
environment and explanation of fundamental concepts of the shell. To enter the
shell in interactive mode run the following command:

\begin{lstlisting}[numbers=none]
$ lmishell
\end{lstlisting}
\funclistend
It is often useful to also append |--verbose| or |--more-verbose| arguments to
the |lmishell| command to get some tracing and debugging information.

\subsubsection{Establishing connection}
\label{openlmi:shell:examples:connection}

Before we start working with the providers we have to establish connection
with a WBEM server.

\begin{lstlisting}[]
$ lmishell 
> conn = connect("<host>", "<username>", "<password>")
\end{lstlisting}
\funclistend
You can run any Python code in LMI shell. Therefore, to test whether the
connection was successful or not we can use one of the following commands:

\begin{lstlisting}[]
> isinstance(conn, LMIConnection)
True
> conn is None
False
\end{lstlisting}
\funclistend
Now when we have successfully established the connection with CIMOM, we can
start querying registered providers. The LMI shell reflects CIM namespace,
classes and object into Python environment and makes those elements accessible
via the instance of |LMIConnection|.

In our examples, the CIM namespace is located at |conn.root.cimv2| path and all
CIM classes registered withing the Object Manager are available under this
namespace -- e. g. class |LMI_Service| can be accesses as
|conn.root.cimv2.LMI_Service|.

We will shorten the namespace part to |ns|, for the purpose of our examples.
Thus the class will be accessible as |ns.LMI_Service|. We can achieve this in
the shell with simple assignment:

\begin{lstlisting}[]
> ns = conn.root.cimv2
\end{lstlisting}
\funclistend
There are several useful built-in functions in the shell that simplifies the
manipulation with CIM objects and their associations.

\subsubsection{Documentation}
\label{openlmi:shell:examples:documentation}

Documentation is a key property of WBEM world, since there are many classes
with a lot of constrained properties and methods. This section introduces
few ways how to obtain useful information.

Shell command |help()| will print description for the LMI shell. If it
is run with object as an argument, it will display verbose information about
the Python (not CIM) object (e. g. |ns.LMI_Service|).

Detailed information about selected CIM class can be displayed by running
|<class>.doc()| command. For example, if we want to see the description of
|LMI_Service|, we will execute |ns.LMI_Service.doc()|.

\subsubsection{Acquiring class instances}
\label{openlmi:shell:examples:instances}

Instances of CIM classes provides information on the managed elements. We have
two simple ways how to transfer these object from WBEM server to client. Those
are the following static methods that are present in each class:

\begin{funcproto}
<class>.instances()
\end{funcproto}
\begin{funcdesc}
This method returns an array of all instances of class |<class>|.
\end{funcdesc}

\begin{funcproto}
<class>.instance_names() 
\end{funcproto}
\begin{funcdesc}
This method is very similar to the first one except it returns a smaller objects
that contain only key properties of given class. It may be used to reduce
data transfer and operation time when you are interested only in a small
amount of instances but you need to filter them out first.
\end{funcdesc}
\funclistend
These methods have an optional parameter of dictionary type that specifies
conditions that the returned instances have to fulfill. It is used to perform
a simple search in the set of instances. The argument is in the form
|{'Property': 'value', ...}|.

Sometimes it is desirable to get only one instance, usually when we search for a
unique combination of properties. For this purpose the previous methods have
also their equivalents that return only one instance and that the first one
found. The methods are:

\begin{funcproto}
<class>.first_instance()
<class>.first_instance_name() 
\end{funcproto}
\funclistend
Sometimes it is enough to work only with key properties of the object and then
the instance name suffice, but it may be simply converted to full instance when
needed with |<instance_name>.to_instance()| method.

\subsubsection{Finding associations}
\label{openlmi:shell:examples:associations}

As described in \cmplref{wbem:cim:is}, associations represents relationships
between objects. There are two ways how to work with these associations. We
can either ask ``What objects are associated with this instance?'' or ``What
associations is this object part of?''. Each instance provides methods that
allow the operator to ask precisely those questions. These methods are:

\begin{funcproto}
<instance>.associators()| 
\end{funcproto}
\begin{funcdesc}
This method returns all objects that are associated with the given object
|<instance>| (i. e. it returns end-points of an association).
\end{funcdesc}

\begin{funcproto}
<instance>.references()| 
\end{funcproto}
\begin{funcdesc}
This method returns all associations that contains reference to the object
|<instance>| (i. e. it returns instances of association classes).
\end{funcdesc}
\funclistend
Similar to instances there are equivalent methods that return only first found
element: |first_associator()| and |first_reference()|.


These methods may take several parameters. One of these parameters is
|ResultClass| which specifies what object types shall be looked up. More
arguments can be found in OpenLMI documentation.

\subsubsection{Managing system services}
\label{openlmi:shell:examples:services}

System services are represented as instances of |LMI_Service| class. This class
provides basic information about the service and methods to change the state of
the service -- i. e. start, stop, enable, disable, etc.

At first we will list names of all services that are currently running on the
system. State of the service is accessible through |Status| property and the
value that symbolizes a correctly running service is |OK|.

\begin{lstlisting}[]
> for service in ns.LMI_Service.instances():
...     if service.Status == "OK":
...             print service.Name
... 
systemd-logind.service
autofs.service
systemd-vconsole-setup.service
systemd-random-seed.service
lvm2-lvmetad.service
abrtd.service
...
\end{lstlisting}
\funclistend
We can also left filtering of the instances on OpenLMI by providing a dictionary
argument to |instances()| or |first_instance()| method. 

\begin{lstlisting}[]
> filter = {'Status': 'OK'}
> for service in ns.LMI_Service.instances(filter):
...     print service.Name
... 
systemd-logind.service
autofs.service
systemd-vconsole-setup.service
systemd-random-seed.service
lvm2-lvmetad.service
abrtd.service
...
> print ns.LMI_Service.first_instance(filter).Name
systemd-logind.service
\end{lstlisting}
\funclistend
Now we will try to restart all services that are in a wrong state.

\begin{lstlisting}[]
> filter = {'Status': 'Error'}
> for service in ns.LMI_Service.instances(filter):
...     print "Restarting service %s" % service.Name
...     service.RestartService()
... 
\end{lstlisting}
\funclistend
The following example will enable |firewalld| service on system startup.

\begin{lstlisting}[]
> filter = {'Name': 'firewalld.service'}
> service = ns.LMI_Service.first_instance(filter)
> service.TurnServiceOn() 
\end{lstlisting}

\subsubsection{Managing accounts}
\label{openlmi:shell:examples:accounts}

Users on the system are represented by |LMI_Account| class and groups are
instances of |LMI_Group|. Every user and group is associated with an instance of
|LMI_Identity| which serves as a container for both users and groups. The
purpose of this class is to have all system identities contained in one place.

The next code will list all users that are present on the system and
whose password is about to expire within the next two days.

\begin{lstlisting}[]
> for user in ns.LMI_Account.instances():
...     if user.PasswordExpirationWarning is not None \
...        and user.PasswordExpirationWarning <= 2:
...             print user.Name
... 
pbrezina 
\end{lstlisting}
\funclistend
The following example will print all groups that user |pbrezina| is a member of.
At first, we need to acquire instance of this user's account. Then we have to
find the identity the user is associated with. Next we use this identity to find
all associated groups.

We use |first_associator()| and |associators()| methods to find desired objects
that are associated with our user and identity respectively. The attribute
|ResultClass| specifies the type of which the end point of our association
should be.

\begin{lstlisting}[]
> filter = {'Name': 'pbrezina'}
> acc = ns.LMI_Account.first_instance(filter)
> ident = acc.first_associator(ResultClass='LMI_Identity')
> for group in ident.associators(ResultClass='LMI_Group'):
...     print group.Name
... 
vmusers
pbrezina
\end{lstlisting}

\section{LMI Scripts}
\label{openlmi:scripts}

OpenLMI defines a Python interface that allows to run any sophisticated code
implementing this interface as one simple command on command line through |lmi|
application.

OpenLMI team develops and maintains a set of scripts that follows this
interface. These scripts aim on simplification of the common use cases of
existing LMI providers thus administrators do not have to deal with Python code,
CIM instances and associations for basic tasks.

The following section introduces few examples of this feature.

\subsection{Examples}
\label{openlmi:scripts:examples}

The first example shows how to print available commands and their usage. There
is one universal command |help| for this purpose.

\begin{lstlisting}[]
$ lmi help
Commands:
  hwinfo         - Display hardware information.
  power          - System power state management.
  service        - System service management.
  user           - POSIX user information and management.

For more informations about particular command type:
    help <command>

$ lmi help service
System service management.

Usage:
    lmi service list [(--enabled | --disabled)]
...
\end{lstlisting}
\funclistend
The next listing contains an example of how to list enabled services and
their state and how to obtain information about a user.

\begin{lstlisting}[]
$ lmi -h <host> --user <user> service list --enabled
password:
Name                      Status
abrtd                     Running
acpid                     Running
auditd                    Running
autofs                    Running
ntpd                      Stopped - OK
...

$ lmi -h <host> --user <user> user show pbrezina
password:
Name     UID Home           Login shell Password last change
pbrezina 529 /home/pbrezina /bin/bash   2013/10/10
\end{lstlisting}
\funclistend
Instead of providing only one host to connect to, we can provide a whole set
of hosts and execute the same command on each host at once. The following
example demonstrates this ability by rebooting all machines listed in file.
\begin{lstlisting}[]
$ cat hosts
host1.example.com
host2.example.com
...
hostN.example.com

$ lmi --hosts-file hosts --user <user> --same-credentials \
>     power reboot
password:
\end{lstlisting}
\funclistend Parameter |--same-credentials| will ensure that the password
provided at the first prompt will be used to authenticate on every host from the
file, instead of asking for each host separately.

