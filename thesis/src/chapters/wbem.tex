\chapter{Web-Based Enterprise Management}
\label{chapter:wbem}

Web-Based Enterprise Management (WBEM)\footnotemark is an open standard
developed by Distributed Management Task Force (DMTF)\footnotemark organization.
It aims to unify management of system resources in distributed computing
environments -- from low level process, network or storage management to higher
level view on system state and configuration of specific components and
provided services.

Every managed element is looked upon as an object with attributes and methods.
WBEM defines how the objects look like, how to transfer their representation
over a network and how to call methods on these objects. It also defines a
standard set of supported resources and provides base templates to help
extending this set with proprietary or open solutions.

To achieve its goal, WBEM widely reuses other DMTF standards, mainly Common
Information Model (CIM) which provides the tools for description of managed
elements as objects and introduces basic concepts of how to work with them. It
is, however, quite loose on implementation details as it does not specify
several key things to make the management really independent on underlying
technologies and software components. For example, it does not specify a
protocol to transfer objects over a network or any specific server architecture.
WBEM is basically a standardized implementation of CIM which eliminates these
weak points by defining new standards like CIM-XML transfer protocol or by
declaring that a server has to be split into a controller and providers that
take care of selected components.

Besides these it also incorporates other standards that help WBEM to fit into
enterprise environments such as Service Location Protocol to locate WBEM servers
inside a corporate network or Uniform Resource Identifier to identify and
distinguish every managed resource.

\footnotetext{http://www.dmtf.org/standards/wbem}
\footnotetext{A standard development organization: http://www.dmtf.org}

\section{Architecture}
\label{wbem:architecture}

Web-Based Enterprise Management standard consist of several key components:
WBEM server and providers, communication protocol and CIM schema. It does not
specify any particular user interface, therefore any kind of command line,
graphical or browser user interface can be used to manage remote resources. The
architecture is illustrated on Figure \ref{fig:wbem-arch}.

\begin{figure}[H]
  \centering
  \tikzstyle{process}=[draw, solid, sharp corners]
\tikzstyle{ui}=[process, minimum height=6cm]
\tikzstyle{server}=[process, minimum height=1.5cm, minimum width=6cm,
    fill=yellow!40]
\tikzstyle{provider}=[process, minimum height=1.5cm, minimum width=2cm,
    fill=green!40]
\tikzstyle{server-group}=[draw, dashed, rounded corners, fill=gray!10]
\tikzstyle{resources}=[process, minimum height=1.5cm, minimum width=6cm,
    fill=blue!30]
\tikzstyle{other}=[process, minimum height=1.5cm, minimum width=2cm,
    fill=green!20]

\begin{tikzpicture}
    \node (server-group) [server-group, matrix] {
        \node (schema) [other, align=center] at (2cm, 3cm) {CIM \\ Schema};
        \node (users) [other, align=center] at (-2cm, 3cm) {Access \\ Control};
        \node (server) [server] at (0,0) {WBEM Server};
        \node (provider1) [provider] at (-2cm, -3cm) {Provider};
        \node (provider2) [provider] at (2cm, -3cm) {Provider};
        \node (provider) at (0, -3cm) {\ldots};
        \\
    };
    
    \node (ui) [ui, align=center] at (-7cm, 0) {User \\ Interface};
    \node (resources) [resources] at (0,-7cm) {Managed System Resources};
    \node (os) [fill=blue!10, align=center, minimum width=6cm] at (0,-5cm)
        {Operating System \\ Interface};
    
    \draw [latex-latex] (server.south) -- (0, -1.5cm) -| (provider1.north);
    \draw [latex-latex] (server.south) -- (0, -1.5cm) -| (provider2.north);
    \draw [latex-latex] (server.west) -- (ui.east)
        node[midway, align=center, fill=red!20] {CIM-XML \\ over HTTP};
    \draw [latex-latex] (provider1.south) -- ($ (resources.north) - (2cm, 0) $);
    \draw [latex-latex] (provider2.south) -- ($ (resources.north) + (2cm, 0) $);
    \draw [latex-latex] (schema.south) -- ($ (server.north) + (2cm, 0) $);
    \draw [latex-latex] (users.south) -- ($ (server.north) - (2cm, 0) $);
\end{tikzpicture}
  \caption{Architecture of WBEM}
  \label{fig:wbem-arch}
\end{figure}

\subsection{Components}

This section briefly describes the fundamental components of WBEM standard
in bottom-up order.

\subsubsection{CIM schema}

CIM schema defines a set of managed objects. It describes how the objects look
like: what attributes do they have, what methods they support and what are the
relationships between those objects. For this purpose Common Information Model
standard and its conceptual object oriented language called Managed Object
Format (MOF) are used. More information on these standards can be found in
Section \ref{wbem:cim}.

\subsubsection{WBEM provider}

Provider is a server module that implements selected components of the CIM
schema. It basically creates a mapping between operating system resources into a
format understood by the server. Many providers are dependent on underlying
operating system, since each system has usually different interface to access
its objects -- especially the low level ones.

\subsubsection{WBEM server}

WBEM server, which is also often called CIM Object Monitor (CIMOM), is the
access point between a client and the providers. It handles user authentication
and authorization and it uses the schema to route commands and data into a
particular provider and to call selected method to complete the operation.
Depending on implementation, it may also contain a simple persistent database to
be used by the providers.

\subsubsection{Provider interface}

The WBEM standard itself does not specify any particular programming interface
between the providers and CIMOM thus basically every implementation defines its
own API. Therefore one provider can not be used across multiple implementations
of WBEM server. Since this proved to be a huge disadvantage in a real world
experience a new standard defining the API emerged under The Open Group
consortium. The API is referred to as Common Manageability Programming Interface
(CMPI) and it is supported by most of the modern CIMOM implementations with the
exception of some proprietary solutions. Providers written for CMPI are mostly
or even completely independent on the server underneath.

For communication between a client and the server the CIM objects and commands
are packed into an XML format (so called CIM-XML) and the payload is being
transferred over a standard HTTP protocol.

\subsection{Workflow}
\label{wbem:architecture:workflow}

At first, the operator will issue a command through selected user interface.
This command is packed into a CIM-XML format and sent over an HTTP protocol to
WBEM server. The server then performs user authentication and checks whether the
user is authorized to perform the desired command. If the access is granted to
the user, the server reads the CIM schema and decides to what provider should be
the command forwarded. The request is then processed by the selected provider,
the result is returned to the server and the server sends it back to the client
-- again using CIM-XML over HTTP. The returned value can be either a simple
value (number, string, boolean, \ldots), some of the predefined error codes
(access denied, operation not supported, invalid arguments, \ldots) or the whole
object described by its attributes.

\section{Common Information Model}
\label{wbem:cim}

Common Information Model (CIM) is an open standard that defines how to represent
managed resources as a set of objects and relationships between them and how to
exchange information about these elements between multiple parties.

CIM is split into two parts: an Infrastructure Specification (\ref{wbem:cim:is})
and CIM schema (\ref{wbem:cim:schema}).
Infrastructure Specification defines architecture and concepts of CIM. The
schema is then a huge set of predefined classes that pursue the needs of
enterprise environments.

\subsection{CIM Infrastructure Specification}
\label{wbem:cim:is}

CIM Infrastructure Specification defines how to describe managed elements using
an object oriented approach. Every element is an instance of some class. A class
defines attributes that provides information on the object and methods that
represent an operation on the object which may result in altering the object
state.

Any relationship between two objects is declared as an association. Association
is a special class that contains references to two or more instances of some
(not necessarily distinct) classes. It can model all one to one, one to many and
many to many relations.

Common Information Model also supports polymorphism, thus any class or
association can be derived from or marked as abstract (i. e. it can not be
instantiated).

The definition language use to describe CIM elements is called Managed Object
Format. The fundamental syntax of this language is described in Section
\ref{wbem:cim:mof}.

\subsection{CIM Schema}
\label{wbem:cim:schema}

CIM Schema is a conceptual model that describes many classes and associations to
cover most of the needs of enterprise environments. The schema is split into
three categories: the Core Schema, Common Schemas and Extension Schemas.

The Core Schema is a base model. It provides classes and associations that are
shareable within all areas of management. Those elements usually serve as a base
classes to be derived from to create more precise description of specific
resources. An example of such class is |CIM_ManagedElement| which provides basic
information about managed element such as its name, description and generation.
This class should be on top of an inheritance tree for all managed elements.

Common Schemas provide more specific classes for common management areas such
as network configuration, processes and services maintenance, storage,
databases, accounts, etc. These classes, however, are designed generically
enough to be independent on any underlying technology and lack any
vendor-specific characteristics. For example |CIM_Account| is used to represent
user accounts and it declares attributes like name, last login time, password
expiration but it does not provide anything system specific like a UNIX login
shell.

The last category Extension Schemas contains vendor specific extensions to the
Common Schemas which became a part of the standard. For example |CIM_UnixFile|
representing a UNIX file or |CIM_J2eeJVM| describing Java Virtual Machine
utilized by a J2EE server.

\subsection{Managed Object Format}
\label{wbem:cim:mof}

Managed Object Format (MOF) is a conceptual object oriented language based upon
UML, that supports many typical features of object oriented languages such as
classes, structures, enumerations and qualifiers (annotations). However, CIM
itself uses only a subset of this language. It supports only qualifiers and
simple class definitions, where attributes can be defined only as a primitive
type or a reference to a class in case of an association.

\subsubsection{Class declaration}
\label{wbem:cim:mof:class}

Class declaration follows a common syntax of modern object oriented languages.
This syntax is enriched by qualifiers which can be applied to every class,
attribute, method or method argument declaration.

\begin{lstlisting}[caption={MOF class declaration},
label=lst:wbem:mof:class]
[qualifiers]
class ClassName : SuperClass
{
    [qualifiers]
    type AttributeName;
    ...
    
    type MethodName([qualifiers] type ArgName, ...);
    ...
};
\end{lstlisting}

Qualifiers are a widely used concept in CIM, since they are not only used as
guidelines for code editors but they main purpose is to specify the nature of
the qualified element -- how it should be interpreted, value range and other
conditions.

For example, an argument may be marked as input or output. Attribute may be
qualified as a |Bitmap|, meaning that every bit of the attribute has some
specific interpretation. Another qualifier called |BitValues| can specify how
the bits shall be interpreted. It may also be used to specify minimum
(|MinValue|) or maximum value (|MaxValue|) of an numeric attribute or argument.
Very common qualifier is |Description| that provides documentation of the class
element. Another important qualifier is |Key| that marks an attribute or a set
of attributes as unique in all existing instances of the same class and it is
used as a part of the instance address.

The following example shows an abstract class representing a person. This class
cannot be instantiated and serves as a base class for classes |Man| and |Woman|.

\begin{lstlisting}[caption={MOF example: Person},
label=lst:wbem:mof:class]
[Abstract,
 Description("Class representing a human beeing.")]
class Person
{
    [Key]
    uint32 NationalIdentificationNumber
    
    string FirstName;
    string LastName;
    
    [MinValue(1), MaxValue(31)]
    uint32 DayOfBirth;
    
    [MinValue(1), MaxValue(12)]
    uint32 MonthOfBirth;
    
    uint32 YearOfBirth;
    
    uint32 ComputeAge();
};

class Man : Person
{

}

class Woman : Person
{
    string BirthLastName;
}
\end{lstlisting}

\subsubsection{Association declaration}
\label{wbem:cim:mof:instance}

Association is declared in the same way as a class with an addition of
|Association| qualifier. This qualifier allows inclusion of references into the
class declaration. References are marked with |REF| keyword. Besides references,
associations can also have attributes and methods in the same way as classes
can.

Cardinality of a relation can be controlled by |Min| and |Max| qualifiers
specifying minimum and maximum cardinality respectively. Those qualifiers are
associated with each reference.

The next example will create an association between previous classes |Man| and
|Woman| symbolizing their marriage. The qualifier |Max(1)| declares the marriage
as monogamy. The association also have an attribute representing number of
children that came up from this marriage.

\begin{lstlisting}[caption={MOF example: association},
label=lst:wbem:mof:association]
[Association,
 Description("Married couples.")]
class MarriedCouples
{
    [Key, Max(1)]
    Man REF Husband;
    
    [Max(1)]
    Women REF Wife;
    
    uint32 NumberOfChildren;
};
\end{lstlisting}

\subsubsection{Instance declaration}
\label{wbem:cim:mof:instance}

It is possible to create an instance of a class or an association directly using
MOF syntax. The following example will create instance of a man and a woman and
marry them.

\begin{lstlisting}[caption={MOF example: instantiation},
label=lst:wbem:mof:instantiation,
morekeywords={instance, of, as}]
instance of Man as $Adam
{
    NationalIdentificationNumber = 8807014142;
    FirstName = "Adam";
    LastName = "Hunt";

    DayOfBirth = 1;
    MonthOfBirth = 7;
    YearOfBirth = 1988;
};

instance of Woman as $Eva
{
    NationalIdentificationNumber = 9053141122;
    FirstName = "Eva";
    LastName = "Hunt";

    DayOfBirth = 14;
    MonthOfBirth = 3;
    YearOfBirth = 1990;
    
    BirthLastName = "Abott";
};

instance of MarriedPersons
{
    Husband = $Adam;
    Wife = $Eva;
    
    NumberOfChildren = 0;
};
\end{lstlisting}

\section{Overview of existing implementations}
\label{wbem:implementations}

Several open source and proprietary projects exist that implement the Web-Based
Enterprise Management on different scale. A small overview of the most notable
ones follows.

\subsection{SBLIM}
\label{wbem:implementations:sblim}

SBLIM (pronounced Sublime; acronym for Standards Based Linux Instrumentation for
Manageability) is a collection of open source system management tools that
enables WBEM on Linux systems. It brings the whole set of components from a
client through a set of providers and C and Java language bindings to a server.

An interesting feature of this project is the CIMOM: Small Footprint CIM
Broker (SFCB) that has a memory footprint less than one megabyte which makes it
possible to run it on small embedded systems.

\subsection{WBEM Services}
\label{wbem:implementations:wbemservices}

WBEM Services is an open source implementation of full WBEM stack written in
Java language. It also brings several supplementary developer tools such as code
generator which converts CIM schema written in MOF into a Java source code,
graphical schema browser and many examples.

\subsection{Windows Management Instrumentation}
\label{wbem:implementations:wmi}

Windows Management Instrumentation (WMI) is a Microsoft's proprietary
implementation. Even though it is highly based on WBEM standard, it is not
completely compatible and it requires providers to be specially written only for
WMI using COM/DCOM API interface since WMI does not support CMPI.

\subsection{OpenLMI}
\label{wbem:implementations:openlmi}

OpenLMI (Open Linux Management Infrastructure) is an emerging open source
project that brings together multiple existing pieces of WBEM stack and also
adds several new tools like an LMI shell with Python based syntax. Besides that
it brings many Linux specific providers and more are about to come. OpenLMI aims
to be a new standard for Linux environments that will redefine current ways of
configuration of system components.
