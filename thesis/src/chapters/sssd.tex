\chapter{System Security Services Daemon}
\label{chapter:sssd}

System Security Services Daemon (SSSD)\footnote{SSSD upstream:
\url{https://fedorahosted.org/sssd}} is a set of background processes that
are responsible for providing and caching identity\footnote{User and groups,
services, automounter maps\ldots}
and policy\footnote{Sudo rules, SSH host keys, \ldots} objects from a central
directory server. Its main goal is to cache this information so it remains
available in an offline mode (when the directory server is not accessible) but
still keep serving up to date data when possible. Since communication with
remote directory server is slow in general, there is always a compromise between
data accuracy and a response speed.

SSSD puts stress on a response speed for publicly accessible data, but tries to
be as much accurate as possible with information that brings up potentional
security-sensitive decisions. For example, identity information such as user
names and group membership are always returned from cache, unless the records
are expired. On the other hand, user authentication is always performed directly
against directory server (or Kerberos) as long as it is reachable. Also those
information remains in the cache until SSSD is positive that it was removed from
database, thus the scenario when a user is unable to login into his or her
laptop because it was offline for a long time and cached data has expired simply
can not happen. Besides many more features, this caching mechanism is one of the
biggest advantage over |nslcd|\footnote{De facto standard daemon that caches
NSS-LDAP results, which is currently being replaced by SSSD on many Linux
distributions.}.

\section{Main features}
\label{sssd:main-features}

SSSD is an open source project with very active upstream development and
community. Since it has already mastered fundamental identity tasks, the current
focus is mainly on the field of Active Directory integration -- or in other
words, using UNIX-based systems as clients in Microsoft Windows domain. SSSD
has many mature features, here is a list of the most important ones:

\begin{itemize}
  \item An open source project with active upstream development and community
  \item More predictable caching mechanism than |nslcd|
  \item In addition to identity information it, also manages services, autofs
        maps, sudo rules, SSH host keys
  \item Integration with Kerberos -- automatic ticket renewal, seemingless
        ticket request during transition to online mode, \ldots
  \item Integration with FreeIPA -- IPA sites, host based access control,
        cross-domain trusts, \ldots
  \item Integration with Active Directory -- AD sites, trusts, SID to POSIX ID
        mapping, evaluation of token groups, \ldots
  \item Dynamic DNS updates 
\end{itemize}

\section{Architecture}
\label{sssd:arch}

SSSD consists of multiple processes running on system level and several client
libraries. The processes are: monitor, one or more data providers and one or
more responders. Graphical representation of all those parts can be seen on
Figure \ref{fig:sssd-arch}.

Monitor can be considered as some sort of SSSD manager. It is responsible for
the life cycle of all the other processes -- it starts them, signals them,
kills them and also reboots them in case of an error.

The task of a data provider is to fetch data from directory server, store them
in cache and keep them up to date. Depanding on the object type, records may
be downloaded either per request triggered by responder or some other outer
event or it can be done periodically in the background. There is one data
provider running for each domain.

Responders represent supported services like NSS\footnotemark user and
group maps, sudo policy, PAM authentication, etc. Each service has its own
responder. A responder listens for client's request and when a request comes,
it searches the cache and report requested records back to the client.

\footnotetext{Name Service Switch, a mechanism that allows to specify different
data sources for standard system services.}

Finally, the client libraries are light-weight libraries that implement
pluggable interface of target application. Its only goal is to be a middle-man
between the application and SSSD responder, i. e. it packs a request into
a protocol implemented by selected responder and then unpacks the reply and
return the response to the caller.

\begin{figure}[H]
  \centering
  \tikzset{%
  cascaded/.style = {%
    general shadow = {%
      shadow scale = 1,
      shadow xshift = -2ex,
      shadow yshift = 2ex,
      draw,
      fill = white},
    general shadow = {%
      shadow scale = 1,
      shadow xshift = -1ex,
      shadow yshift = 1ex,
      draw,
      fill = white},
    fill = white, 
    draw}}
    
\tikzstyle{database}=[
    cylinder,
    cylinder uses custom fill,
    cylinder body fill=gray!20,
    cylinder end fill=gray!10,
    shape border rotate=90,
    aspect=0.25,
    draw,
    minimum height=4em
]
\tikzstyle{process}=[draw, solid, sharp corners]
\tikzstyle{process-node}=[minimum height=2em, minimum width=4em,
    text height=1.5ex, text depth=.25ex, fill=yellow!10]
\tikzstyle{process-group}=[draw, dashed, rounded corners, fill=blue!10]
\tikzstyle{dp-process}=[process, cascaded, fill=gray!10]
\tikzstyle{dp-module}=[process, process-node]
\tikzstyle{resp-process}=[process, process-node]
\tikzstyle{monitor-process}=[process, process-node, fill=red!30]
\tikzstyle{client-process}=[process, fill=gray!10]
\tikzstyle{client-module}=[process, process-node]

\tikzstyle{sensor}=[draw, fill=blue!20, text width=5em, 
    text centered, minimum height=2.5em]
\tikzstyle{ann} = [above, text width=5em]
\tikzstyle{naveqs} = [sensor, text width=6em, fill=red!20, 
    minimum height=12em, rounded corners]

% Define distances for bordering
\def\blockdist{2.3}
\def\edgedist{2.5}

\begin{tikzpicture}
    \node (monitor) [monitor-process] {Monitor};

    \node (dp) [dp-process, matrix, below of=monitor, below=3em, anchor=north] {
        \node (id) [dp-module] {Id};
        \node (auth) [dp-module, right of=id, anchor=west] {Auth};
        \node (chpass) [dp-module, right of=auth, anchor=west] {Chpass};
        \node (access) [dp-module, right of=chpass, anchor=west] {Access};
        \node (sudo) [dp-module, below of=id] {Sudo};
        \node (autofs) [dp-module, below of=auth] {Autofs};
        \node (center) [right of=autofs, anchor=center] {};
        \node (subdom) [dp-module, below of=chpass] {Subdom};
        \node (host) [dp-module, below of=access] {Host};
        \node (dp-name) [below of=center, anchor=center] {Data Provider}; \\
    };
    
    \node (responders) [process-group, matrix, below of=dp, below=5em,
                        anchor=north] {
        \node (nss) [resp-process] {nss};
        \node (pam) [resp-process, right of=nss, anchor=west] {pam};
        \node (sudo) [resp-process, right of=pam, anchor=west] {sudo};
        \node (center) [right of=sudo, anchor=center] {};
        \node (autofs) [resp-process, right of=sudo, anchor=west] {autofs};
        \node (ssh) [resp-process, right of=autofs, anchor=west] {ssh};
        \node (pac) [resp-process, right of=ssh, anchor=west] {pac};
        \node [below of=center, anchor=center] {Responders}; \\
    };

    \node (clients) [client-process, matrix, below of=responders, below=4em,
        anchor=north] {
        \node (libs) [client-module] {client library};
        \node [below of=libs, anchor=north]
            {NSS, PAM, \ldots}; \\
    };
    
    \node (mc) [process, below of=responders, below right=2em, anchor=west,
        fill=green!30]
        {Fast in-memory cache};
    
    \node (ldap) [database, right of=monitor, right=11em] {LDAP};
    \path let \p1 = (dp), \p2 = (ldap) in
        node (sysdb) [database, cylinder body fill=yellow!50, cylinder end
        fill=yellow!30] at (\x2,\y1) {Cache};
    
    \draw [latex-latex] (responders) -- (dp);
    \draw [dotted, -latex] (monitor) -- (dp);
    \draw [dotted, -latex]
        (monitor) -| ($ (responders.north west) + (2em, 0) $);
    \draw [latex-latex] ($ (dp.north east) - (2em, 0) $) |- (ldap);
    \draw [latex-latex] (dp.east) -- (sysdb);
    \draw [latex-] (responders.east) -| (sysdb);
    \draw [latex-latex] (responders.south) -- (libs.north);
    \draw [latex-latex] (libs.east) -| (mc.south);
\end{tikzpicture}

  \caption{Architecture of SSSD}
  \label{fig:sssd-arch}
\end{figure}

Different parts of SSSD use different inter-process communication methods.
Monitor communicates with other processes either via signals or it uses D-Bus
for more sophisticated messages. Communication between responders and data
providers is performed through D-Bus protocol only. On the other hand,
communication of client libraries and responders is done via sockets to keep
client libraries as light-weight as possible.

\section{Cache management}
\label{sssd:cache}

As was already said, the cache is populated by data providers and consumed by
responders. This is due to asynchronous nature of SSSD -- one process can still
serve data to other clients while simultaneously the other process stores new or
updates existing records.

SSSD has implemented several caching techniques that aims to improve the ratio
between data consistency and response speed. Since every object type requires a
little bit of individual approach, not all mechanisms are always used. This
section will introduce all methods from basic expiration time and negative cache
through mid-point refresh and all the way to sophisticated out of band and
background periodic updates.

\subsection{Record expiration}
\label{sssd:cache:expiration}

Record expiration is the most basic approach to cache management. Simply put,
each entry in the cache has associated a timestamp after which is the entry
marked as expired -- but not invalid. When a client requests an expired object,
the further actions depends on immediate accessibility of directory server. If
SSSD is in online mode -- it can connect to the server -- it will first update
the object and then return it. However, if the server is unreachable, SSSD will
simply return what is contained within the cache.

Expired records are only deleted from the cache if SSSD will not find them on
the directory server. Thus as long as SSSD is in offline mode, they will remain
accessible and usable.

\subsection{Negative cache}
\label{sssd:cache:negative}

Negative cache is a special in-memory cache for negative results. If the
requested record is not found, it is stored in the negative cache for a short
time span, making all successive requests that falls into this period as fast
as possible.

Some special users and groups are stored in this cache permanently for the whole
SSSD life time. This is usually used to filter out local users and groups which
are not handled by SSSD. For example the |root| account is present in the
negative cache by default. Therefore any request for information about |root|
quickly shortcuts to ``entry not found'' result, making the caller try the next
NSS |passwd| source.

\subsection{Mid-point refresh}
\label{sssd:cache:midpoint}

Mid-point refresh is an out of band refresh in the middle of the expiration
period. If a request is raised before the record is expired but after a
configured percent of its expiration time is reached, it is returned
immediately. However, at the same time it is also being refreshed at the
background.

This mechanism guarantees a cache-hit for objects that are being frequently
requested, making the queries fast, without contacting any remote server.

\subsection{Periodic updates}
\label{sssd:cache:periodic}

Periodic updates are events triggered periodically on the background by data
provider. They purpose is to update a larger set of records at once or to
download all new records that are not yet contained within the cache.

These updates usually uses |entryUSN|\footnotemark attribute or last modification
timestamp of an entry to detect those objects that were modified or newly added
on the server since the last update was performed.

Currently the following periodic updates are implemented: enumeration, refresh
of records that are about to expire and refresh of sudo rules. The last one is
described in its own section (\cmplref{sssd:cache:sudo}).

\footnotetext{Special attribute supported by modern directory servers. It
contains a unique number that represents modification sequence of records
stored on the server. Id est record with lower entryUSN value is older than
record with higher value.}

\subsubsection{Enumeration}
\label{sssd:cache:periodic:enum}

Enumeration is a special task that simply stores and keeps up to date all
records of standard NSS maps like users, groups and services. It is used to make
enumeration commands (those that list all or many of NSS objects at once) fast
-- for example listing |/home| directory or running |getent passwd| command. If
all users had to be yet downloaded from LDAP server, it could take very long
time to finish, depending on the size of the environment. This task will
pre-fetch them, making those commands much faster -- from several minutes (or
even tens of minutes) to a few seconds.

This, however, comes with a price. Periodic updates of thousands of users and
groups and they membership evaluation triggers significant amount of network
traffic and processor time on both server and client side. Thus enumeration is
disabled in SSSD by default and administrators are in general discouraged from
using it unless it is really necessary.

\subsubsection{Nearly expired records}
\label{sssd:cache:periodic:expired}

This is a very specific periodic update. Its intention is to keep all records in
the cache up to date in such way that these records never expire. It takes all
records that are about to expire within the next period and re-downloads them.

At the time of this thesis, this refresh type is only implemented for netgroups.
Its purpose is to reduce user login time in large enterprise environments with
thousands of netgroups organised in a complex nested topology. Without this
update, a user signing in may trigger a refresh of all these expired netgroups
which may take several minutes. That is unacceptable, enabling this update
reduces login time back to seconds.

\subsection{Fast in-memory cache}
\label{sssd:cache:memory}

In addition to standard SSSD cache, users and groups also use small in-memory
cache that is accessible directly by client libraries. This cache may be
understood as L1-L3 cache memories used in modern processors.

If client library finds requested object in in-memory cache, it returns it
directly without communicating with proper responder. Records stored in
this cache have a relatively short lifetime -- only few minutes -- after that
the records are invalid and need to be fetched again from the responder.

The purpose of this cache is to make listing commands as fast as possible,
since it often queries same user or group over and over again thus cutting down
the communication between client library and responder to necessary minimum is
a nice benefit.

\subsection{Caching of sudo rules}
\label{sssd:cache:sudo}

Sudo rules are very specific objects from the point of SSSD. They creates a
potential security risk since you do not want to allow a user to use sudo unless
he is really supposed to. However at the same time it is not possible to go
directly to a server every time sudo is used as it is done with user
authentication, since the nature of sudo is that there are usually several
successive sudo commands in a short time thus it is simply not acceptable to
wait several seconds at each command -- a cache has to be used.

This makes caching of sudo rules quite a challenge because they are supposed to
be as much up to date as possible but also keeping good responsiveness of sudo
command. For this purpose a combination of periodic and event based updates
was designed. These updates are called full refresh, smart refresh, rules
refresh and out of band full refresh of sudo rules. Each of them has different
purpose and their explanation follows.

\subsubsection{Full refresh}
\label{sssd:cache:sudo:full}

Full refresh is a periodic task and does exactly what its name suggest -- it
downloads a complete image of sudo rules stored on the server and replaces
everything that is contained withing the cache with new content. This refresh is
used mainly to detect rules that were removed from the server and make sure
that those rules are also deleted from the cache.

Possible disadvantage is that it may produce a non-trivial network traffic,
depanding on the size of the environment, thus it is supposed to have a long
period -- by default 6 hours is used.

\subsubsection{Smart refresh}
\label{sssd:cache:sudo:smart}

Smart refresh is another periodic update. It is responsible for keeping current
rules stored in the cache up to date and expanding the cache with newly added
rules. It uses |entryUSN| attribute or last modification timestamp if the first
one is not present to detect those modified or new rules.

Unlike full refresh, smart refresh never deletes any rules from the cache. And
since it checks only for a difference it usually creates only a negligible 
network usage so it is possible to run this refresh type frequently. 

\subsubsection{Rules refresh}
\label{sssd:cache:sudo:rules}

This type corresponds to basic record expiration. When a user wants to perform a
sudo command, sudo responder retrieves all rules that can be possibly applied to
this user and issues an update request for all expired rules from this set.
Purpose of this refresh is to ensure that SSSD does not allow any unauthorized
access (i. e. when a rule was removed from the server).

\subsubsection{Out of band full refresh}
\label{sssd:cache:sudo:oob}

Sudo rules are usually very stable, they are rarely modified and rarely deleted.
Thus when SSSD detects that one rule has been deleted, it takes it as a sign
that more changes could have been done on the server. Therefore when the rules
refresh ends up with removing a rule from the cache an out of band full refresh
is triggered ensuring the cache consistency.

\subsubsection{Summary of sudo refresh types}
\label{sssd:cache:sudo:summary}

All sudo refresh methods are usually used altogether to provide the best
possible experience when running sudo with SSSD as its data source. However, it
is important for administrators to understand how they work separately so they
can configure the daemon appropriately for needs of their environment. The Table
\ref{table:sssd-sudo-refresh} compares the most important features of those
refresh types.

\begin{table}[H]
  \centering
  \begin{tabular}{l l}
\multicolumn{2}{c}{\textbf{When it is triggered (by default)}} \\
\hline\hline
Full refresh & every six hours or when a rule is deleted from the cache \\
\hline
Smart refresh & every fifteen minutes \\
\hline
Rules refresh & when user runs sudo, rules expires after ninety minutes \\
\hline
\multicolumn{2}{c}{} \\
\multicolumn{2}{c}{\textbf{What is its purpose}} \\
\hline\hline
Full refresh & to keep the cache consistent \\
\hline
Smart refresh & store new a update modified rules \\
\hline
Rules refresh & do not grant user more privilege \\
\hline
\multicolumn{2}{c}{} \\
\multicolumn{2}{c}{\textbf{What operations does it perform}} \\
\hline\hline
Full refresh & insert, delete \\
\hline
Smart refresh & insert, modify \\
\hline
Rules refresh & modify, delete \\
\hline
\end{tabular}

  \caption{Comparison of sudo rules refresh methods}
  \label{table:sssd-sudo-refresh}
\end{table}

\section{Data Providers}
\label{sssd:providers}

Main purpose of data provider is to populate cache with required data and keep
those information as much up to date as it is possible. Another and not any less
important task is to perform user authentication directly against remote server.
It is the most privileged process in SSSD since it is the only one that is
permitted to write data into the cache and it is also the only part of the
daemon that is actually allowed to communicate with remote resources like Active
Directory, IPA, legacy directory server, Kerberos server, etc.

Data provider internally consist of several modules. Each module is responsible
for different task. For example identity module manages user accounts and
groups, authentication module performs authentication, access module checks if a
user have enough permissions to sign into a local machine, sudo module caches
sudo rules, etc.

Every module can use different data source, or back-end in SSSD's terminology.
The back-end usually specifies type of directory server used: either legacy
LDAP, IPA or Active Directory. However, some modules also allows a special
back-ends, especially those that deals with user authorization. For example
authentication module can be configured either to perform bind against directory
server or to acquire Kerberos ticket. Access provider can use advanced LDAP
based access control list or simple user and group white list or black list.

\section{Responders}
\label{sssd:responders}

The key task of responder processes is to satisfy all requests from clients with
corresponding information stored within the cache. Compared to data providers
where each process represents one domain and maintains data for all of its
supported services, responders works in exactly opposite direction. There exists
one responder for every enabled SSSD service (NSS, PAM, sudo, etc.), that reads
the service-related data from multiple domains.

The multi-domain search works as a first match method. Responder iterates
through domains and the first domain that contains the requested object is used
unless a fully qualified object name is provided (i. e. both object name and
domain name) -- in that case only the given domain is searched. Listing
\ref{lst:sssd:responder} shows a pseudocode for multi-domain search of user by
name.

\begin{lstlisting}[caption={Multi-domain lookup (get user by name)},
morekeywords={end, fi, each},label=lst:sssd:responder]
responder receives a request containing username
for each domain
    user := lookup user by username in domain;
    if user is not found or is expired
        let data provider of this domain look it up;
        user := lookup user by username in domain;
        if user is not found
            continue;
        fi
    fi

    return user;
end for
\end{lstlisting}

It is worth to note that responder has the cache opened only in read-only mode
and it never contacts any remote server. If it finds out that some record should
be refreshed or that it does not have all information that is required to
satisfy client's request it triggers an update process in data provider of
selected domain. When the update is finished it reads the cache again for new
data.

\section{Client applications and libraries}
\label{sssd:clients}

Client applications are third party applications, consumers of SSSD managed
data. They usually provide a pluggable interface that is used to extend their
functionality or to choose different data source. SSSD implements this interface
in a client library that is then dynamically loaded by target programs. The
library is supposed to be as much light-weight as possible to minimize necessary
overhead, therefore it tries to be very small and does not use any additional
dependencies unless they are really required.

The goal of such library is to be a middle-man between SSSD responder and the
consumer. Some of the client libraries (like |libsss_nss|) implements additional
level of caching, but the main purpose is only to communicate with proper
responder to obtain requested data and transfer this information in a format
that is understood by destined application.

When compared to communication inside SSSD where a D-Bus is used to transfer
commands and data, to speed thing up the communication between client library
and responder is performed via a socket, using a wired protocol appropriately
selected for the type of transferred data.
