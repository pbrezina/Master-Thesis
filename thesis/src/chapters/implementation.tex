\chapter{Implementation of OpenLMI SSSD provider}

The goal of this thesis is to implement an SSSD provider for the OpenLMI
project. The provider is supposed to export an interface to configure the most
fundamental parts of the SSSD daemon and to retrieve basic information about
managed domains. The required features of the provider are:

\begin{itemize}
  \item enable and disable domains and services
  \item changing debug level of SSSD components
  \item provide basic information about managed domains
\end{itemize}

This chapter will first describe designed CIM schema for the SSSD, which
specifies the provider interface. Then it will briefly introduce several
considered but abandoned solutions and at last it will describe the final
implementation design in detail.

\section{CIM Schema for OpenLMI SSSD provider}

The first step to creation of a new OpenLMI (or CIM in general) provider is to
create a CIM schema -- a conceptual model of the managed element. The model is
described in Managed Object Format language (see section \cmplref{wbem:cim:mof}
for details).

Even though the MOF is an object oriented language, it is not recommended to use
the same concepts and techniques to design classes that represents the managed
objects as we are used to from object oriented programming languages. In
programming languages, we usually hide the complexity of an object in private
properties and methods and then we export a public interface that provides a
higher level model of this object so it is simple to use.

However, CIM schema goes the other way around. It is supposed to define a very
low level model and reflect the managed elements and their relationships as
accurate as possible. This often means that there are no methods of type
|GetObjects()| to retrieve all instances of specific objects. Also there are
usually no setter for the property, the instance should be modified directly.

The following definition is a small part of my very first iteration of the SSSD
CIM schema, which wrongly follows my experience from programming languages.
It shows the difference between CIM concepts and OOP.

\begin{lstlisting}[morekeywords={}]
class LMI_SSSD
{
    [Static]
    uint32 GetServices([Out] LMI_SSSD_Service REF services[]);
    
    [Static]
    uint32 GetBackends([Out] LMI_SSSD_Backend REF backends[]);
};
\end{lstlisting}

\funclistend There are several issues with this model. At first, every CIM class
is supposed to inherit from one of the existing classes -- either the most
generic |CIM_ManagedElement| or any other more concrete class that fits our
needs.

This looks like a reasonable requirement, however these classes defines a lot of
additional properties and methods that our class inherits, but we most certainly
do not need all of them. It is an unfortunate, but common practice that the
provider for the new class does not implement most the inherited elements.

The second issue is that the class contains methods that should be either
completely omitted \footnotemark or if some relationship exist between SSSD and
services/back-ends it is supposed to be implemented as an association.

\footnotetext{Since we can do the same by fetching all instances of
LMI_SSSD_Service and LMI_SSSD_Backend classes.
See \cmplref{openlmi:shell:examples:instances}}

The last issue is just a triviality. It breaks a naming convention, since
managed services should be named with |Service| suffix.

The correct way of modelling such objects that follows the CIM concepts is:

\begin{lstlisting}
class LMI_SSSDService : CIM_Service
{

};

[Abstract, Description("Base class for SSSD's components.")]
class LMI_SSSDComponent : CIM_ManagedElement
{
    /* common properties and methods */
};

/* The SSSDService is taken so we had to choose an
 * equivalent word from SSSD terminology. */ 
class LMI_SSSDResponder : LMI_SSSDComponent
{
    /* responder specific properties and methods */
};

class LMI_SSSDBackend : LMI_SSSDComponent
{
    /* back-end specific properties and methods */
};

[Association]
class LMI_SSSDAvailableResponders
{
    [Key, Min(1), Max(1)]
    LMI_SSSDService REF SSSD;
    
    [Key]
    LMI_SSSDResponder REF Responder;
};

[Association]
class LMI_SSSDAvailableBackends
{
    [Key, Min(1), Max(1)]
    LMI_SSSDService REF SSSD;
    
    [Key]
    LMI_SSSDBackend REF Backend;
};
\end{lstlisting}

\subsection{Final schema}

This section will present an UML diagram of the final version of the SSSD CIM
schema and describe its important design decisions and components. This version
was accepted by OpenLMI developers. The full MOF file can be found in Appendix
\cmplref{appendix-mof}.

\section{Considered solutions}

\section{D-Bus interface}

\section{OpenLMI scripts}